\documentclass[a4paper,11pt]{book}


% usepackage
\usepackage{amsmath, amssymb, amsfonts, latexsym, mathtools}
\usepackage[amsmath,amsthm,thmmarks]{ntheorem}
\usepackage{thmtools}
\usepackage{mdframed} %for theorem frame
\usepackage{physics}
\usepackage[colorlinks=false,citecolor=black,linkcolor=black]{hyperref}%
\usepackage{tikz-cd}
\usepackage{tikz}
\usepackage{tikz-3dplot}
\usetikzlibrary{shapes.geometric}
\usetikzlibrary{intersections,calc,arrows.meta}
\usepackage{url}
\usepackage{bm} % for vector bold
\usepackage{amscd} % For diagram
\usepackage{cases}% case 
\usepackage{cancel}% canceling
\usepackage{subfig}
\usepackage{thm-restate}
\usepackage{dynkin-diagrams}
\usepackage{pgfkeys, pgfopts}
\usepackage{ytableau}
\usepackage{array} % for table width
\usepackage{hhline} % double horizontal line in table https://ftp.yz.yamagata-u.ac.jp/pub/CTAN/macros/latex/required/tools/hhline.pdf
\usepackage{xcolor}
\usepackage{pagecolor,lipsum} % http://ctan.org/pkg/{pagecolor,lipsum}

\begin{document}

\title{An Introduction to Linear Algebra}
\author{Noda Shuto}
\date{\today}
\maketitle
Let's write an article ...

\url{https://mathlog.info/articles/1403}

Let $ k \geq 2 $ be a natural number.
We define a sequence $ \left\{ a_n \right\} $ by
\begin{equation*}
  \begin{cases}
    a_{n+k} = a_{n+k-1} + a_{n+k-2} + \cdots + a_{n} = \displaystyle \sum_{i=n}^{n+k-1}a_{i} \\
    a_0 = a_1 = \cdots = a_{k-2} = 0 \\
    a_{k-1} = 1 .
  \end{cases}
\end{equation*}
We call $ \left\{ a_n \right\}  $ a generalized Fibonacci sequence for $ k $.  




\begin{thebibliography}{99}
\bibitem{textbook} author, \emph{title of a book}, publisher, year
\end{thebibliography}
\end{document}